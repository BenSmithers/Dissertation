\documentclass[main.tex]{subfiles}

%\externalcitedocument{bibfile}

\begin{document}


\section{The Ghosts of the Real World}
The three-mass and three active-flavor neutrino \index{neutrino} paradigm has been well-studied~\cite{PhysRevD.98.030001,Esteban_2019,de_Salas_2018,Capozzi_2016,zboson2006, berns2021recent}.
However, several anomalies persist at short baselines, including in $\nu_\mu\rightarrow\nu_e $ appearance in decay-in-flight~\cite{aguilar2018significant} and decay-at-rest~\cite{Athanassopoulos_1998} beams  and $\nu_e\rightarrow\nu_e$ disappearance at reactors~\cite{mention2011reactor,serebrov2019first}  and with $^{71}$Ga electron capture sources~\cite{PhysRevC.73.045805,giunti2011statistical}.  
These anomalies have been attributed to possible oscillations of unknown neutrinos with mass-squared differences in the range of $\Delta m^{2}\sim 0.1-10\text{ eV}^{2}$~\cite{abazajian2012light}.   
Such an additional neutrino flavor state must be non-weakly interacting, or ``sterile,'' to be consistent with observed decay widths of the Z-boson~\cite{zboson2006}; the simplest such model is known as the ``3+1'' light sterile neutrino model in which a single sterile neutrino is added. 

There have been interesting recent developments for the 3+1 model.  
The BEST experiment appears to validate the anomalous electron neutrino disappearance signature of the previous gallium anomalies with a new level of statistical significance and experimental precision~\cite{barinov2021results}. 
The Neutrino-4 experiment claims evidence of short-baseline oscillations in the $\bar{\nu}_e$ disappearance channel with $\Delta m^2\sim 7.3\,\mathrm{eV}^2$ at the 2.9$\sigma$ level.
 Meanwhile results from the MicroBooNE~\cite{microboonecollaboration2021search,microboonecollaboration2021search1,microboonecollaboration2021searchmulti} experiment challenge the interpretation that the MiniBooNE low energy excess~\cite{miniboone2018} is due entirely to the electron neutrino by placing a constraint on the sterile neutrino interpretation of the excess; though the impact of this observation on the 3+1 model has yet to be assessed.  Continued exploration of sterile neutrino mixing in all channels and all energy ranges thus remains strongly motivated~\cite{sbnfermilab}.

The addition of a fourth neutrino mass and flavor eigenstate expands the unitary mixing matrix to four dimensions. 
The four-neutrino oscillations model becomes an extension of the three-neutrino model with three additional mixing angles $\theta_{14}$, $\theta_{24}$, and $\theta_{34}$, and two new CP-violating phases $\delta_{14}$ and $\delta_{24}$. These three new mixing angles parametrize the amplitude of oscillations between the three active states and the one sterile state, and lead to additional short-baseline vacuum-like oscillations as well as novel effects in the presence of matter~\cite{Akhmedov:1988kd,KRASTEV1989341,Chizhov:1998ug, Chizhov_1999, Akhmedov_2000, Nunokawa:2003ep,Petcov:2016iiu}.  In this work we consider CP-conserving models with all CP-violating phases set to zero.

Of particular interest to neutrino telescopes, matter effects can result in the near complete disappearance of TeV-scale muon anti-neutrinos passing through the Earth's core for a sterile neutrino with eV-scale mass squared differences~\cite{Nunokawa:2003ep, Choubey:2007ji, Barger:2011rc, Esmaili:2012nz, esmaili2013restricting, Lindner:2015iaa}. This signature of matter-enhanced resonant disappearance has been targeted by the IceCube Neutrino Observatory~\cite{Aartsen_2020, Aartsen_2020_prd}, leading to one of the  most sensitive $\nu_\mu$ disappearance analyses to date. The result of the analysis was a closed 90\% contour with best fit point at $\sin^2 2\theta_{24}\sim0.1$ and $\Delta m^2_{14}=4.5\text{ eV}^2$, under a conservative assumption (for the $\nu_\mu$ disappearance channel) that $\theta_{34}=\theta_{14}=0$. In addition to being a strong refutation, lower mass solutions consistent with the LSND~\cite{Athanassopoulos_1998} and MiniBooNE anomalies and constraints around 1~eV$^2$~\cite{kopp2013sterile, Cirelli:2004cz, abazajian2012light, Gariazzo:2017fdh, Dentler:2017tkw, Diaz:2019fwt}, a possible interpretation of this result is as a statistically weak hint of a disappearance signature around $\Delta m^2_{41}\sim4.5\text{ eV}^2$.  Further exploration of this region of parameter space  in other channels at neutrino telescopes is therefore strongly motivated. 

\subsection{Three-Neutrino oscillations}
\index{neutrino!oscillations}

We first consdier there being three 

\begin{equation}
    U_{\text{PMNS}} = \left(\begin{array}{ccc} U_{e1} & U_{e2} & U_{e3} \\ U_{\mu 1} & U_{\mu 2} & U_{\mu 3} \\ U_{\tau 1} & U_{\tau 2} & U_{\tau 3} \end{array}\right)
\end{equation}
Such that, 
\begin{equation}
    \left(\begin{array}{ccc} \nu_{e} & \nu_{\mu} & \nu_{\tau} \end{array}\right)  = \left(\begin{array}{ccc} U_{e1} & U_{e2} & U_{e3} \\ U_{\mu 1} & U_{\mu 2} & U_{\mu 3} \\ U_{\tau 1} & U_{\tau 2} & U_{\tau 3} \end{array}\right) \left(\begin{array}{c} \nu_{1} \\ \nu_{2} \\ \nu_{3} \end{array}\right).
\end{equation}
Considering a neutrino originally in the electron flavor state and at energy $E$, then propagated over a distance \(L\), we can calcualte the odds of measuring it in the same electron flavor state. This is the $P(\nu_{e}\to\nu_{e})$ survival probability. Expressed in the mass eigenstate, 
\begin{equation}
    \ket{\nu_{e}} = U_{e1} \ket{\nu_{1}} + U_{e2} \ket{\nu_{2}} + U_{e3}\ket{\nu_{3}},
\end{equation}
and so in order to consider the propagation of these mass eigenstates, we consider the time-dependent Schr\"odinger equation
\begin{equation}
i\dfrac{\partial}{\partial t} \ket{\nu_{i} (t)} = \bvec{H}_{\nu}\ket{\nu_{i}(t)},
\end{equation}
which can be solved with the stationary state solution 
\begin{equation}
    \ket{\nu_{i} (t)}  =  e^{-iEt} \ket{\nu_{i} (0)}.
\end{equation}
Neutrino mass is therefore relevant, and since the neutrino mass is so small we can make the approximation  
\begin{align}
E_{i} &= \sqrt{p^{2}c^{2} + m_{i}^{2}c^{4}} \\
&\approxeq pc + \dfrac{m_{i}^{2}c^{4}}{2E_{i}}
\end{align}
So then applying this to the above setup 

\begin{equation}
    \ket{\nu_{i} (t)}  =  e^{-ipt}e^{ m_{i}^{2}/(2E}\ket{\nu_{i} (0)}
\end{equation}
and 
\begin{equation}
\ket{\nu_{e}(t)} = \sum\limits_{i} U_{ei} e^{-ipt}e^{ m_{i}^{2}/(2E)}\ket{\nu_{i} (0)}
\end{equation}

\subsection{3+1 Neutrino Oscillations}
\index{neutrino!oscillations}
An additional fourth neutrino mass an flavor eigenstate will modify exxpected oscillations. To parametrize this, three additional mixing angles and two additional CP-violating phases must be added to the standard three neutrino paradigm. The new four-neutrino unitary mixing matrix becomes.
\begin{equation}
    U_{\text{PMNS}} = \left(\begin{array}{cccc} 1 & 0 & 0 & 0 \\ 0&\cos\theta_{23}&\sin\theta_{23}&0 \\ 0&-\sin\theta_{23}&\cos\theta_{23}&0 \\ 0&0&0&1 \end{array}\right)
\end{equation}

\subsection{Neutrino Interactions}
\index{neutrino!interactions}
Discuss neutrino interactions at medium to high energies. Deep Inelastic Scattering

\subsection{Matter Effects on Neutrion Oscillations}
Words on the matter effect. \index{neutrino!matter effect} Effective neutrino masses. 




\end{document}