\documentclass[main.tex]{subfiles}

\externalcitedocument{bibfile}

\begin{document}

\section{The Ghosts of the Real World}
The three-mass and three active-flavor neutrino paradigm has been well-studied~\cite{PhysRevD.98.030001,Esteban_2019,de_Salas_2018,Capozzi_2016,zboson2006, berns2021recent}.
However, several anomalies persist at short baselines, including in $\nu_\mu\rightarrow\nu_e $ appearance in decay-in-flight~\cite{aguilar2018significant} and decay-at-rest~\cite{Athanassopoulos_1998} beams  and $\nu_e\rightarrow\nu_e$ disappearance at reactors~\cite{mention2011reactor,serebrov2019first}  and with $^{71}$Ga electron capture sources~\cite{PhysRevC.73.045805,giunti2011statistical}.  
These anomalies have been attributed to possible oscillations of unknown neutrinos with mass-squared differences in the range of $\Delta m^{2}\sim 0.1-10\text{ eV}^{2}$~\cite{abazajian2012light}.   
Such an additional neutrino flavor state must be non-weakly interacting, or ``sterile,'' to be consistent with observed decay widths of the Z-boson~\cite{zboson2006}; the simplest such model is known as the ``3+1'' light sterile neutrino model in which a single sterile neutrino is added. 

There have been interesting recent developments for the 3+1 model.  
The BEST experiment appears to validate the anomalous electron neutrino disappearance signature of the previous gallium anomalies with a new level of statistical significance and experimental precision~\cite{barinov2021results}. 
The Neutrino-4 experiment claims evidence of short-baseline oscillations in the $\bar{\nu}_e$ disappearance channel with $\Delta m^2\sim 7.3\,\mathrm{eV}^2$ at the 2.9$\sigma$ level.
 Meanwhile results from the MicroBooNE~\cite{microboonecollaboration2021search,microboonecollaboration2021search1,microboonecollaboration2021searchmulti} experiment challenge the interpretation that the MiniBooNE low energy excess~\cite{miniboone2018} is due entirely to the electron neutrino by placing a constraint on the sterile neutrino interpretation of the excess; though the impact of this observation on the 3+1 model has yet to be assessed.  Continued exploration of sterile neutrino mixing in all channels and all energy ranges thus remains strongly motivated~\cite{sbnfermilab}.

The addition of a fourth neutrino mass and flavor eigenstate expands the unitary mixing matrix to four dimensions. 
The four-neutrino oscillations model becomes an extension of the three-neutrino model with three additional mixing angles $\theta_{14}$, $\theta_{24}$, and $\theta_{34}$, and two new CP-violating phases $\delta_{14}$ and $\delta_{24}$. These three new mixing angles parametrize the amplitude of oscillations between the three active states and the one sterile state, and lead to additional short-baseline vacuum-like oscillations as well as novel effects in the presence of matter~\cite{Akhmedov:1988kd,KRASTEV1989341,Chizhov:1998ug, Chizhov_1999, Akhmedov_2000, Nunokawa:2003ep,Petcov:2016iiu}.  In this work we consider CP-conserving models with all CP-violating phases set to zero.

Of particular interest to neutrino telescopes, matter effects can result in the near complete disappearance of TeV-scale muon anti-neutrinos passing through the Earth's core for a sterile neutrino with eV-scale mass squared differences~\cite{Nunokawa:2003ep, Choubey:2007ji, Barger:2011rc, Esmaili:2012nz, esmaili2013restricting, Lindner:2015iaa}. This signature of matter-enhanced resonant disappearance has been targeted by the IceCube Neutrino Observatory~\cite{Aartsen_2020, Aartsen_2020_prd}, leading to one of the  most sensitive $\nu_\mu$ disappearance analyses to date. The result of the analysis was a closed 90\% contour with best fit point at $\sin^2 2\theta_{24}\sim0.1$ and $\Delta m^2_{14}=4.5\text{ eV}^2$, under a conservative assumption (for the $\nu_\mu$ disappearance channel) that $\theta_{34}=\theta_{14}=0$. In addition to being a strong refutation, lower mass solutions consistent with the LSND~\cite{Athanassopoulos_1998} and MiniBooNE anomalies and constraints around 1~eV$^2$~\cite{kopp2013sterile, Cirelli:2004cz, abazajian2012light, Gariazzo:2017fdh, Dentler:2017tkw, Diaz:2019fwt}, a possible interpretation of this result is as a statistically weak hint of a disappearance signature around $\Delta m^2_{41}\sim4.5\text{ eV}^2$.  Further exploration of this region of parameter space  in other channels at neutrino telescopes is therefore strongly motivated. 

\subsection{Three-Neutrino oscillations}

Neutrino oscillations are fundamentally facilitated through a PMNS matrix allowing us to re-express neutrino flavor states in a mass basis.

\begin{equation}
    U_{\text{PMNS}} = \left(\begin{array}{ccc} U_{e1} & U_{e2} & U_{e3} \\ U_{\mu 1} & U_{\mu 2} & U_{\mu 3} \\ U_{\tau 1} & U_{\tau 2} & U_{\tau 3} \end{array}\right)
\end{equation}
Such that, 
\begin{equation}
    \left(\begin{array}{ccc} \nu_{e} & \nu_{\mu} & \nu_{\tau} \end{array}\right)  = \left(\begin{array}{ccc} U_{e1} & U_{e2} & U_{e3} \\ U_{\mu 1} & U_{\mu 2} & U_{\mu 3} \\ U_{\tau 1} & U_{\tau 2} & U_{\tau 3} \end{array}\right) \left(\begin{array}{c} \nu_{1} \\ \nu_{2} \\ \nu_{3} \end{array}\right).
\end{equation}
Considering a neutrino originally in the electron flavor state and at energy $E$, then propagated over a distance \(L\), we can calcualte the odds of measuring it in the same electron flavor state. This is the $P(\nu_{e}\to\nu_{e})$ survival probability. Expressed in the mass eigenstate, 
\begin{equation}
    \ket{\nu_{e}} = U_{e1} \ket{\nu_{1}} + U_{e2} \ket{\nu_{2}} + U_{e3}\ket{\nu_{3}},
\end{equation}
and so in order to consider the propagation of these mass eigenstates, we consider the time-dependent Schr\"odinger equation
\begin{equation}
i\dfrac{\partial}{\partial t} \ket{\nu_{i} (t)} = \bvec{H}_{\nu}\ket{\nu_{i}(t)},
\end{equation}
which can be solved with the stationary state solution 
\begin{equation}
    \ket{\nu_{i} (t)}  =  e^{-iEt} \ket{\nu_{i} (0)}.
\end{equation}
From here, the neutrino masses enter the equation
\begin{align}
E_{i} &= \sqrt{p^{2}c^{2} + m_{i}^{2}c^{4}} \\
&\approxeq pc + \dfrac{m_{i}^{2}c^{4}}{2E_{i}}
\end{align}
So then applying this to the above setup 

\begin{equation}
    \ket{\nu_{i} (t)}  =  e^{-ipt}e^{ m_{i}^{2}/(2E}\ket{\nu_{i} (0)}
\end{equation}
and 
\begin{equation}
\ket{\nu_{e}(t)} = \sum\limits_{i} U_{ei} e^{-ipt}e^{ m_{i}^{2}/(2E)}\ket{\nu_{i} (0)}
\end{equation}


\subsection{3+1 Neutrino Oscillations}
An additional fourth neutrino mass an flavor eigenstate will modify exxpected oscillations. To parametrize this, three additional mixing angles and two additional CP-violating phases must be added to the standard three neutrino paradigm. The new four-neutrino unitary mixing matrix becomes.
\begin{equation}
    U_{\text{PMNS}} = \left(\begin{array}{ccc} 1 & 0 & 0 & 0 \\ 0&\cos\theta_{23}&\sin\theta_{23}&0 \\ 0&-\sin\theta_{23}&\cos\theta_{23}&0 \\ 0&0&0&1 \end{array}\right)
\end{equation}

\section{IceCube}

The IceCube Neutrino Observatory is described at length in Ref.~\cite{Aartsen_2017}. Briefly, the detector is a cubic-kilometer Cherenkov neutrino observatory one and half kilometers deep in the Antarctic ice~\cite{Aartsen_2017}.
There, 5160 photo-multiplier tubes encased within glass pressure vessels, or ``Digital Optical Modules'' (DOMs)~\cite{ABBASI2009294} detect Cherenkov emission from charged particles traversing the ice.
The DOMs are arranged vertically with a seventeen meter spacing into seventy-nine strings, which themselves are aligned into a hexagonal lattice with a 125 meter spacing. 
An additional, more densely instrumented sub-detector called DeepCore exists towards the bottom-center of the main detector~\cite{ABBASI2012615}.
The observatory has been running for over a decade and has accumulated large numbers of $\nu_{\mu}$ CC interactions which make depositions of light that make long signatures in the detector called tracks; and neutral current, electron neutrino, and tau neutrino events which deposit light in blob-like shapes called cascades. These event topologies are elaborated upon in Section. %~\ref{sec:depo}.

\section{Event topologies}

Large-volume neutrino telescopes typically are sensitive in the TeV to PeV energies; here, Deep-Inelastic Scattering (DIS)~\cite{gandhineutrinos} and the recently-observed~\cite{IceCube:2021rpz} Glashow-Resonance~\cite{PhysRev.118.316} interactions dominate. 
The detected neutrino interaction events fall into two morphological categories: tracks and cascades.

Charged-current (CC) $\nu_{\mu}$ DIS events result in muons at energies where radiative processes dominate energy loss rates.
As a result, energy losses are stochastically driven and the produced muons travel for kilometers. 
The results are threefold: muons are difficult to fully contain in neutrino telescopes, muon energies are poorly correlated with progenitor muon-neutrino energies, and muons' long travel-distance can allow for reconstructing their direction to within 1$^{\circ}$~\cite{trackaccuracy2017}. These events are called \textit{tracks}~\cite{icecube_energy_reco}.

All neutral-current (NC) DIS events result in a hadronic shower spreading around the interaction point and a secondary neutrino invisibly carrying away a proportion of the parent neutrino's energy. 
These events are often contained with a spherical topology. 
$\nu_{e}$-CC interactions develop similarly to neutral-current interactions, but repeated inverse Compton scattering of the produced electron initiates an electromagnetic shower superimposed over the hadronic shower. 
Thus, nearly all of the interacting neutrino's energy is observable as detectable light. 
These events are called \textit{cascades.} Such events tend to be well-contained permitting an efficient energy reconstruction, although suffer from poor angular reconstruction~\cite{icecube_energy_reco}. 

The evolution of a $\nu_{\tau}$-CC interaction is highly dependent on the energies involved. A tau is produced simultaneously as a hadronic cascade propagates around the interaction point, and then the tau decays. 
Due to their large mass, taus have a short lifetime and a decay length of $\sim 50$ m per PeV of tau energy~\cite{abbasi2020measurement}. 
From the tau branching ratios~\cite{PhysRevD.98.030001}, 17.37\% of the charged tau decays evolve as muon tracks, while the remainder of the decays evolve as electromagnetic or hadronic cascades. Only at neutrino energies above 60 TeV do $\nu_{\tau}$-CC interactions yield events with distinguishable primary and secondary cascades~\cite{abbasi2020measurement}.

Several distinct event samples have been developed to study these different types of events in IceCube. The High-Energy Starting Events sample~\cite{2021hese}, for example, was developed to study both taus and high-energy neutrinos likely astrophysical in origin. There exist other events samples optimized for higher event rates at lower energies, such as the Medium-Energy Starting Events~\cite{PhysRevDoverone}, and the five-year inelasticity sample~\cite{inelasticity2019}. There are also samples optimized for muon purity, such as the eight-year atmospheric muon sample~\cite{Aartsen_2020_prd} and others optimized for accurate energy resolution such as the six-years cascade sample~\cite{sixyrscascade}. This work will consider the cascade event selection described in~\cite{2018PhDT17N} and the track event selection previously used in IceCube sterile neutrino searches~\cite{PhysRevLett.117.071801}.


\end{document}