\documentclass[main.tex]{subfiles}

%\externalcitedocument{bibfile}

\begin{document}


\section{Neutrinos}
Three of the seventeen fundamental particles, known to exist, are neutrinos.~\ref{fig:party}
\begin{figure}
    \centering
    \includegraphics[width=0.8\linewidth]{figures/particles.png}
    \caption{A table of the seventeen fundamental particles. Only the mass-eigenstates for the fermions are shown.}\label{fig:party}
\end{figure}
This three-mass and three active-flavor neutrino \index{neutrino} paradigm has been well-studied~\cite{PhysRevD.98.030001,Esteban_2019,de_Salas_2018,Capozzi_2016,zboson2006, berns2021recent}, and is the conventional understanding.
At least two neutrinos are known to be massive, but as of the time of writing there exist only upper bounds on their masses, and it is uncertain by which mechanism neutrinos get their mass.

However, several anomalies persist at short baselines, including in $\nu_\mu\rightarrow\nu_e $ appearance in decay-in-flight~\cite{aguilar2018significant} and decay-at-rest~\cite{Athanassopoulos_1998} beams  and $\nu_e\rightarrow\nu_e$ disappearance at reactors~\cite{mention2011reactor,serebrov2019first}  and with $^{71}$Ga electron capture sources~\cite{PhysRevC.73.045805,giunti2011statistical}.  
These anomalies have been attributed to possible oscillations of unknown neutrinos with mass-squared differences in the range of $\Delta m^{2}\sim 0.1-10\text{ eV}^{2}$~\cite{abazajian2012light}.   
Such an additional neutrino flavor state must be non-weakly interacting, or ``sterile,'' to be consistent with observed decay widths of the Z-boson~\cite{zboson2006}; the simplest such model is known as the ``3+1'' light sterile neutrino model in which a single sterile neutrino is added. 

There have been interesting recent developments for the 3+1 model.  
The BEST experiment appears to validate the anomalous electron neutrino disappearance signature of the previous gallium anomalies with a new level of statistical significance and experimental precision~\cite{barinov2021results}. 
The Neutrino-4 experiment claims evidence of short-baseline oscillations in the $\bar{\nu}_e$ disappearance channel with $\Delta m^2\sim 7.3\,\mathrm{eV}^2$ at the 2.9$\sigma$ level.
 Meanwhile results from the MicroBooNE~\cite{microboonecollaboration2021search,microboonecollaboration2021search1,microboonecollaboration2021searchmulti} experiment challenge the interpretation that the MiniBooNE low energy excess~\cite{miniboone2018} is due entirely to the electron neutrino by placing a constraint on the sterile neutrino interpretation of the excess; though the impact of this observation on the 3+1 model has yet to be assessed.  Continued exploration of sterile neutrino mixing in all channels and all energy ranges thus remains strongly motivated~\cite{sbnfermilab}.

The addition of a fourth neutrino mass and flavor eigenstate expands the unitary mixing matrix to four dimensions. 
The four-neutrino oscillations model becomes an extension of the three-neutrino model with three additional mixing angles $\theta_{14}$, $\theta_{24}$, and $\theta_{34}$, and two new CP-violating phases $\delta_{14}$ and $\delta_{24}$. These three new mixing angles parametrize the amplitude of oscillations between the three active states and the one sterile state, and lead to additional short-baseline vacuum-like oscillations as well as novel effects in the presence of matter~\cite{Akhmedov:1988kd,KRASTEV1989341,Chizhov:1998ug, Chizhov_1999, Akhmedov_2000, Nunokawa:2003ep,Petcov:2016iiu}.  In this work we consider CP-conserving models with all CP-violating phases set to zero.

Of particular interest to neutrino telescopes, matter effects can result in the near complete disappearance of TeV-scale muon anti-neutrinos passing through the Earth's core for a sterile neutrino with eV-scale mass squared differences~\cite{Nunokawa:2003ep, Choubey:2007ji, Barger:2011rc, Esmaili:2012nz, esmaili2013restricting, Lindner:2015iaa}. This signature of matter-enhanced resonant disappearance has been targeted by the IceCube Neutrino Observatory~\cite{Aartsen_2020, Aartsen_2020_prd}, leading to one of the  most sensitive $\nu_\mu$ disappearance analyses to date. The result of the analysis was a closed 90\% contour with best fit point at $\sin^2 2\theta_{24}\sim0.1$ and $\Delta m^2_{14}=4.5\text{ eV}^2$, under a conservative assumption (for the $\nu_\mu$ disappearance channel) that $\theta_{34}=\theta_{14}=0$. In addition to being a strong refutation, lower mass solutions consistent with the LSND~\cite{Athanassopoulos_1998} and MiniBooNE anomalies and constraints around 1~eV$^2$~\cite{kopp2013sterile, Cirelli:2004cz, abazajian2012light, Gariazzo:2017fdh, Dentler:2017tkw, Diaz:2019fwt}, a possible interpretation of this result is as a statistically weak hint of a disappearance signature around $\Delta m^2_{41}\sim4.5\text{ eV}^2$.  Further exploration of this region of parameter space  in other channels at neutrino telescopes is therefore strongly motivated. 

\subsection{Neutrino oscillations}
\index{neutrino!oscillations}

Neutrinos were, initially, thought to be massless neutral particles. 

The weak interaction couples to the SU(2) doublets,
\begin{align}
&\left(\begin{array}{c} \nu_{e} \\ e^{-} \end{array}\right) & &\left(\begin{array}{c} \nu_{\mu} \\ \mu^{-} \end{array}\right) & &\left(\begin{array}{c} \nu_{\tau} \\ \tau^{-} \end{array}\right),
\end{align}
neutrinos will themselves be produced in one of these three flavor-eigenstates.
These flavor eigenstates can be expressed in the mass-basis as a linear superposition of the mass states. 
The relations are encoded in the unitary\footnote{neutrino number needs to be conserved.} PMNS matrix, 
\begin{equation}
    U_{\text{PMNS}} = \left(\begin{array}{ccc} U_{e1} & U_{e2} & U_{e3} \\ U_{\mu 1} & U_{\mu 2} & U_{\mu 3} \\ U_{\tau 1} & U_{\tau 2} & U_{\tau 3} \end{array}\right),
\end{equation}
with which we can write
\begin{equation}
    \left(\begin{array}{ccc} \nu_{e} & \nu_{\mu} & \nu_{\tau} \end{array}\right)  = \left(\begin{array}{ccc} U_{e1} & U_{e2} & U_{e3} \\ U_{\mu 1} & U_{\mu 2} & U_{\mu 3} \\ U_{\tau 1} & U_{\tau 2} & U_{\tau 3} \end{array}\right) \left(\begin{array}{c} \nu_{1} \\ \nu_{2} \\ \nu_{3} \end{array}\right).
\end{equation}
where $\nu_{i}$, $i\in\left(1,2,3\right)$ is in the mass basis and $i\in\left(e,\mu\tau\right)$ is in the flavor basis. 
We can, in general then, express a stationary state in the flavor basis in terms of the mass basis. 

First, let us consider two neutrino oscillations.
A generic SU(2) matrix can be written as 
\begin{equation}
    U_{2d}=\left(\begin{array}{cc} \cos\theta & \sin\theta \\ -\sin\theta & \cos\theta \end{array}\right)
\end{equation}
and so we can write an initial $\ket{\nu_{e}}$ state as 
\begin{equation}
    \ket{\nu_{e}} = \cos\theta \ket{\nu_{1}} + \sin\theta \ket{\nu_{2}} 
\end{equation}
To consider a propagating neutrino, we solve the time-dependent Schr\"odinger Equation
\begin{equation}
    i\dfrac{\partial}{\partial t} \ket{\nu_{i} (t)} = \bvec{H}_{\nu}\ket{\nu_{i}(t)},
\end{equation}
with a stationary state solution
\begin{equation}\label{eq:stationary}
    \ket{\nu_{i} (t)}  =  e^{-iEt} \ket{\nu_{i} (0)}.
\end{equation}
Neutrinos are known to have very small masses~\cite{KATRIN:2021uub}, and so expand out the energy term and cut off terms with $\mathcal{O}(>m^{2})$. 
\begin{align}
    E_{i} &= \sqrt{p^{2}c^{2} + m_{i}^{2}c^{4}} \\
    &\approxeq pc + \dfrac{m_{i}^{2}c^{4}}{2E}
\end{align}
The stationary state solution Eq~\eqref{eq:stationary} becomes, using $t=L/c$ and $c=1$ 
\begin{equation}
    \ket{\nu_{i} (t)}  =  e^{-ipt}e^{ -im_{i}^{2}L/2E}\ket{\nu_{i} (0)}
\end{equation}
and for the state described in Equation~\eqref{eq:nunu},
\begin{equation}
    \ket{\nu_{e}(t)} = e^{-ipt}e^{ -im_{1}^{2}L/2E} \cos\theta \ket{\nu_{1}}  + e^{-ipt}e^{ -im_{2}^{2}L/2E} \sin\theta \ket{\nu_{2}}.
\end{equation}
Considering the muon flavor state expressed in the mass basis,
\begin{equation}
    \bra{\nu_{\mu}} = -\sin\theta\bra{\nu_{1}} + \cos\theta\bra{\nu_{2}},
\end{equation}
Oscillations probabilities can be calculated by projecting the final considered state onto this propagated state.
So, we consider the probability of measuring the neutrino in the muon flavor state some time $t$ after preparing it the electron flavor state. 
\begin{align}
    \braket{\nu_{\mu} | \nu_{e}(t)} &= -e^{-ipt}e^{ -im_{1}^{2}L/2E} \cos\theta\sin\theta  + e^{-ipt}e^{ -im_{2}^{2}L/2E} \sin\theta\cos\theta \\
    \braket{\nu_{\mu} | \nu_{e}(t)} &= \tfrac{1}{2}e^{-ipt}\left[e^{ -im_{2}^{2}L/2E} - e^{ -im_{1}^{2}L/2E}\right] \sin 2\theta
\end{align}
To calculate actual transmission probabilities, we need the conjugate-square of this, $P_{e\mu} \equiv \left|\braket{\nu_{\mu}|\nu_{e}(t)}\right|^{2}$. While the phase term, $e^{-ipt}$, cancels we are left with 
\begin{align}
    \braket{\nu_{\mu} | \nu_{e}(t)} &=\tfrac{1}{4} \left[e^{ -im_{2}^{2}L/2E} - e^{ -im_{1}^{2}L/2E}\right]\left[e^{ im_{2}^{2}L/2E} - e^{ im_{1}^{2}L/2E}\right] \sin^{2} 2\theta \\
    \braket{\nu_{\mu} | \nu_{e}(t)} &=\tfrac{1}{4}\left[1 - e^{-i\Delta m_{21}^{2}L/2E}- e^{i\Delta m_{21}^{2}L/2E}  +1\right]\sin^{2} 2\theta\\
    \braket{\nu_{\mu} | \nu_{e}(t)} &=\tfrac{1}{4}\left[2- 2\cos\left( \Delta m_{21}^{2}L/2E\right) \right]\sin^{2} 2\theta \\
    \braket{\nu_{\mu} | \nu_{e}(t)} &=\left[\dfrac{1-\cos\left( 2\Delta m_{21}^{2}L/4E\right)}{2} \right]\sin^{2} 2\theta \\
    \braket{\nu_{\mu} | \nu_{e}(t)} &=\sin^{2}\left( \Delta m_{21}^{2}L/4E\right) \sin^{2} 2\theta
\end{align}

Neutrino oscillation probabilities are dependent not on the mass of the neutrino eigenstates, but the differences of square of the masses over which oscillations are considered.
Through oscillations we can only measure the absolute of the difference of the masses, and so the ordering of the masses is invisible to neutrino oscillations. 
Since oscillations today are well-established, at minimum two of the neutrino masses must be non-zero. 
Oscillations will also depend on the energy $E$ of the neutrinos involved and the baseline $L$ over which they travel. 

For oscillations over a $L/E$ scale, the mass-squared splitting will affect the frequency of the neutrino oscillations and the elements in the neutrino mixing matrix will affect the amplitudes of the oscillations. \index{neutrino!baseline}

A toy-model example is shown in Figure~\ref{fig:toy_osc}, where an initial $\nu_{\mu}$ flux is propagated over a fixed baseline. 
A varying amount of the initial flux is expected to oscillate to $\nu_{e}$ as a function of energy. 

\begin{figure}
    \centering
    \includegraphics[width=0.6\linewidth]{figures/oscillations_example.png}
    \caption{A toy 2-neutrino oscillations example. The initial flux is shown as a solid blue line, and the propagated fluxes are shown as dashed lines in orange and blue.}\label{fig:toy_osc}
\end{figure}

We next consider the general case of an electron initially in the electron-flavor neutrino state. 
\begin{equation}\label{eq:three}
    \ket{\nu_{e}} =\sum\limits_{j} U_{ej} \ket{\nu_{j}} 
\end{equation}
Much like before we evolve an arbitrary stationary state
\begin{equation}
    \ket{\nu_{j} (t)}  = e^{-ipt}e^{ -m_{j}^{2}L/2E}\ket{\nu_{j} (0)}
\end{equation}
and so the state described in Equation~\eqref{eq:three},
\begin{equation}
    \ket{\nu_{e}(t)} = \sum\limits_{j} U_{ej} e^{-ipt}e^{ -m_{j}^{2}L/2E}\ket{\nu_{j} (0)}
\end{equation}
Again, consider the probability of measuring the neutrino in the muon flavor state some time $t$ after preparing it the electron flavor state. 
\begin{equation}
    \braket{\nu_{\mu} | \nu_{e}(t)} = e^{-ipt} \sum\limits_{j} U_{\mu j}^{*}U_{ej} e^{ -m_{j}^{2}L/2E}
\end{equation}
We need the conjugate-square of this, $P_{e\mu} \equiv \left|\braket{\nu_{\mu}|\nu_{e}(t)}\right|^{2}$.
\begin{equation}\begin{split}
P_{\mu e}&= \sum\limits_{i}\left[\left|U_{\mu i}\right|^{2}\left|U_{e i}\right|^{2} \right.\\
&\hspace{1cm} + 2\sum\limits_{i>j}U_{\mu i}^{*}U_{e i}U_{\mu j}U_{e j}^{*}\left.e^{-i\Delta m_{ij}^{2}L/2E} \right]
\end{split}\end{equation} 
Using the sinusoidal form of the exponential, we can write this in other components 
\begin{equation}\begin{split}
    P_{\mu e}&= \sum\limits_{i}\left[\left|U_{\mu i}\right|^{2}\left|U_{e i}\right|^{2} \right.\\
    &\hspace{1cm} -2\sum\limits_{i>j} \Re(U_{\mu i}^{*}U_{e i}U_{\mu j}U_{e j}^{*})\cos\left(\Delta m_{ij}^{2}L/2E \right) \\
    &\hspace{1cm} -2\sum\limits_{i>j}\Im(U_{\mu i}^{*}U_{e i}U_{\mu j}U_{e j}^{*})\sin\left(\Delta m_{ij}^{2}L/2E \right)
\end{split}\end{equation} 

\subsection{Unitary Matrix Construction for SU(N)}

One can consider that the 3x3 PMNS matrix is embedded in a larger, unitary, NxN mixing matrix.
This could be the case if neutrino oscillation experiment find evidence that the PMNS matrix is non-unitary.

In these cases we construct arbitrary $U\in SU(N)$ as a product of complex rotations parametrized by angles $\theta_{ij}$ and phases $\delta_{ij}$, 
\begin{equation}
    U(\theta_{ij}, \delta_{ij}) = R_{N-1,N}R_{N-2,N}\ldots R_{45}R_{34}R_{24}R_{15}R_{34}R_{24}R_{14}R_{23}R_{13}R_{12}
\end{equation}
as discussed in Ref~\cite{Arguelles:2020hss}

\subsection{Neutrino Interactions}
\index{neutrino!interactions}
Discuss neutrino interactions at medium to high energies. Deep Inelastic Scattering


\section{Passage of Particles through Matter}

\subsection{Cherenkov Radiation}

High enough energy particles passing through matter are capable of moving faster than light in the medium. 

\begin{equation}
    E^{2} = m^{2} + p^{2}
\end{equation}

we can substitute in the relativistic formulation of momentum
\begin{equation}
    p = \dfrac{mv}{\sqrt{1-v^{2}}}
\end{equation}

and so the energy-velocity relation takes the form 
\begin{equation}
    E^{2} = m^{2} + \dfrac{m^{2}v^{2}}{1-v^{2}}
\end{equation}
Consider a material with index of refraction $n = 1./v_{critical}$, where $v$ is the phase velocity light in the medium. 
Particles travelling at this speed above which particles will emit Cherenkov radiation. 
The associated energy $E_{critical}$ can be solved for algebraically,
\begin{align}
        E^{2}_{critical} &= m^{2} + \dfrac{m^{2}v_{critical}^{2}}{1-v_{critical}^{2}} \\
        E^{2}_{critical}  - m^{2} - E^{2}_{critical}v_{critical}^{2}  + m^{2}v_{critical}^{2} &= m^{2}v_{critical}^{2} \\
        E^{2}_{critical}\left(1  -v_{critical}^{2}\right) &=  m^{2} \\
        \sqrt{\dfrac{m^{2}}{\left(1  -n^{-2}\right)}} &= E_{critical} 
\end{align}

Like a sonic boom of light, the Cherenkov light is emitted in a cone at an angle $\theta_{c}$ relative to the direction a particle traveling at velocity $v$ is moving, and is given by
\begin{equation}
\tan\theta_{c} = \sqrt{v^{2}n^{2} - 1} 
\end{equation}
Although the light is emitted in the forward direction, the Cherenkov light wave-front counter-intuitively trails in a cone opening behind the radiating particle. 

The rate of energy loss 
\begin{equation}
-\left(\dfrac{dE}{dx}\right) =\left(2\pi e\right)^{2}\int \left(1-\dfrac{1}{\left(vn\right)^{2}}\right)\nu d\nu
\end{equation}


\subsection{Matter Effects on Neutrion Oscillations}
\index{neutrino!matter effect}
Although neutrino-anything cross sections are small compared to all other standard-model interactions, neutrinos at high energy passing through a large amount of media can result in a measureable effect on the neutrino flux. 
The neutrinos can experience a coherent forward elastic scattering with electrons and nucleons in the medium through which they propagate. Although all three neutrinos can scatter via Z$^{0}$ boson exchange with nucleons; only $\nu_{e}$ can scatter via the exchange of W$^{\pm}$ and Z$^{0}$ with electrons. 
The consequence of these interactions is that neutrinos behave as if they had slightly different masses, which are referred to as effective masses, and so their oscillations are impacted.  
The mass-modification is parametrized through a weak-field effect from the $Z^{0}$ and $W^{\pm}$ on the neutrino flavor state $\nu_{i}$, which we first write out as
\begin{align} 
    V_{\nu_{i}, e}^{Z^{0}+W^{\pm}} &= -\dfrac{\sqrt{2}}{2}G_{F} N_{e} + \sqrt{2}G_{F}N_{e}  & V_{\nu_{i}, n}^{Z^{0}} &= -\dfrac{\sqrt{2}}{2}G_{F} N_{n} & V_{\nu_{i}, p}^{Z^{0}} &= \dfrac{\sqrt{2}}{2}G_{F} N_{p}
\end{align}
where $G_{F}$ is the Fermi constant and $N_{\alpha}$ is the number density of particle $\alpha$ in the medium. 
If we assume similar numbers of protons and neutrons in a medium the effects of neutrons and protons cancel for each neutrino flavor. 
The resulting weak-field effects, per-flavor, are
\begin{align} 
    V_{\nu_{e}}^{Z^{0}+W^{\pm}} &= \dfrac{\sqrt{2}}{2}G_{F} N_{e}  & V_{\nu_{\mu}}^{Z^{0}} &= -\dfrac{\sqrt{2}}{2}G_{F} N_{e} & V_{\nu_{\tau}}^{Z^{0}} &= -\dfrac{\sqrt{2}}{2}G_{F} N_{e}
\end{align}
Differences in the propagation of the different flavors will appear due to a difference in the weak-field potential. 
Since there are only two unique terms, it follows that 
\begin{equation}
    V \equiv V_{\nu_{e}}^{Z^{0}+W^{\pm}} -  V_{\nu_{\mu}}^{Z^{0}}  =  V_{\nu_{e}}^{Z^{0}+W^{\pm}} -  V_{\nu_{\tau}}^{Z^{0}}   = \sqrt{2} G_{F} N_{e}
\end{equation}
We now define a function $V(\vec{x})\equiv  \sqrt{2} G_{F} N_{e}(\vec{x})$ that depends on the electron number-density at arbitrary position $\vec{x}$.  
Since the electron flavor is the only one with a unique matter-effect, the potential difference that could modify neutrino oscillations goes like 
\begin{equation}
    V = V(x)\left(\begin{array}{ccc} 1&0&0\\0&0&0 \\0&0&0 \end{array}\right) 
\end{equation}

From here, we consider the two-neutrino case and use the small-mass approximation of the neutrino mass
\begin{align} 
    \braket{\nu_{\alpha} | H_{vac} | \nu_{\beta} } &= \braket{ \sum_{i} U_{\alpha i} \nu_{i} \middle| H_{vac} \middle| \sum_{j} U_{\beta j}^{*}\nu_{j}}  \\
    &=  \sum\limits_{j} U_{\alpha j} U^{*}_{e\beta}\left( p + \dfrac{m_{j}^{2}}{2E} \right)
\end{align}
As in the case of vacuum oscillations, it is the relative phases of the mass-eigenstate wave packets that contribute to the interference in the flavor basis. 
The \textit{difference} energies for the different mass states gives rise to neutrino oscillations, so alternatively we can express each of the masses as differences from $m_{1}^{2}$ 
\begin{align}
    H_{\alpha \beta, vac } &= \sum\limits_{j} U_{\alpha j} U^{*}_{\alpha j} \Delta m^{2}_{j1}\\
    H_{vac }&= \dfrac{1}{2E} U\left(\begin{array}{ccc} 0 & 0 & 0 \\ 0 & \Delta m_{21}^{2} & 0 \\ 0 & 0 & \Delta m_{31}^{2} \end{array}\right)U^{\dag}
\end{align}

And so we can construct a modified Hamiltonian with the matter contribution 
\begin{equation}
    H_{tot }= \dfrac{1}{2E} \left[ U\left(\begin{array}{ccc} 0 & 0 & 0 \\ 0 & \Delta m_{21}^{2} & 0 \\ 0 & 0 & \Delta m_{31}^{2} \end{array}\right)U^{\dag} + V(x)\left(\begin{array}{ccc} 1&0&0\\0&0&0 \\0&0&0 \end{array}\right)  \right]
\end{equation}

A common tactic for working with this modified Hamiltonian is in the diagonalized basis, where the new diagonale elements are treated as \textit{effective} mass eigenstates. 


\subsection{Neutrino Propagation with Interactions}
In addition to neutirno oscillations, several other phenomena effect the full description of the neutrino flux as it propagates through the Earth. 
These include, but are not limited to, neutrino flux attenuation as neutrinos interact in the Earth, the process where charged tau leptons produced through neutrino-nucleon interactions decay to produce lower energy tau neutrinos called tau regeneration, and Glashow resonance interactions~\cite{PhysRev.118.316}.  

This section  describes the methods of determining propagation of neutrino fluxes from Earth-surface fluxes to in-IceCube fluxes. 
Here, we summarize the procedures described also in Reference~\cite{arguelles2021nusquids}.

The neutrino flux is described as a function of energy $E$ and location $x$ for neutrino flavor $\alpha$ using the density matrix formalism in the weak-interaction flavor-eigenstate basis as 
\begin{equation}
    \rho(E,x) = \sum_{\alpha} \phi_{\alpha}(E,x) \ket{\nu_{\alpha}}\bra{\nu_{\alpha}}.
\end{equation}
where $\phi_{\alpha}$ specifies the neutrino flux of the flavor $\alpha$. 
The evolution of the system is described by the von Neumann equation
\begin{equation}\label{eq:evol}
    \dfrac{\partial\rho (E,x)}{\partial x} = -i\left[ H(E,x), \rho(E,x)\right],
\end{equation}
where $H$ is the Hamiltonian for the whole system. 
$H$ can be approximated, in the case of small perturbations, as
\begin{equation}
    H(E,x) = H_{0}(E) + H_{1}(E,x)
\end{equation}
where $H_{0}$ is the term giving vacuum neutrino oscillations, which can be solved exactly, and $H_{1}$ is an additional term incorporating matter-effects.
These are, for neutrinos,
\begin{align}
    H_{0}(E) &= \dfrac{1}{2E} \left(\begin{array}{ccc} 0 & 0 & 0\\ 0 & \Delta m_{21}^{2} & 0 \\ 0 & 0 & \Delta m_{31}^{2}\end{array}\right)  \\
    H_{1}(E,x) &= \sqrt{2}G_{F} N_{e}(x) U_{PMNS}^{\dag} \left(\begin{array}{ccc}1&0&0 \\ 0 &0 & 0 \\ 0 & 0 & 0 \end{array}\right) U_{PMNS},
\end{align}
where $G_{F}$ is the Fermi constant, $U_{PMNS}$ is the PMNS neutrino mixing matrix, $N_{e}$ is the electron number density at position $x$, and the $\Delta m^{2}$ terms are the mass-squared splittings.  
Since the evolution of the vacuum component can be soled analytically, it is more convenient to evolve of the system in the interaction basis. 
So, we transform the density matrix and the mass-effect terms as 
\begin{align}
    \rho_{1}(E,x) &= e^{-i H_{0}x}\rho(E,x) e^{-iH_{0}x}. \\
    H_{I,1}(E,x)&= e^{-i H_{0}x} H_{1}(E,x) e^{-iH_{0}x}.
\end{align}
Similarly to Equation~\eqref{eq:evol}, the evolution in the interaction basis is dependent on the matter-effect term,
\begin{equation}\label{eq:mattermod}
    \dfrac{\partial_{1}\rho (E,x)}{\partial x} = -i\left[ H_{I,1}(E,x), \rho(E,x)\right].
\end{equation}
We require a series of additional terms that modify Eq~\eqref{eq:mattermod} to account for the effects which do not preserve neutrino number and energy. 
The first of which is the attenuation of the fluxes from neutrino-Earth interactions, which follow 
\begin{align}\label{eq:nu_evol}
    \Gamma(E,x) &= \dfrac{1}{2}\sum\limits_{\alpha\in(e,\mu,\tau)} \dfrac{\Pi_{\alpha}(E,x) }{\lambda_{NC}^{\alpha}(E,x) + \lambda_{CC}^{\alpha} (E,x)} \\
    \bar{\Gamma}(E,x) &= \dfrac{1}{2}\sum_{\alpha\in(e,\mu,\tau)} \dfrac{\bar{\Pi}_{\alpha(E,x)}}{\bar{\lambda}_{NC}^{\alpha}(E,x) + \bar{\lambda}_{CC}^{\alpha}(E,x) + \bar{\lambda}_{GR}^{\alpha}(E,x) }\label{eq:nubar_evol}
\end{align}
where $\Pi_{\alpha}$ is a neutrino projector onto the flavor $\alpha\in\left\lbrace e,\mu\tau\right\rbrace$, $\nu_{CC}^{\alpha}$ ($\nu_{NC}^{\alpha}$) is the charged (neutral) current neutrino interaction length given by $1/\left[ N_{nuc}(x)\sigma^{\alpha}_{CC(NC)}(E) \right]$~\cite{Formaggio:2013kya, Gandhi:1995tf, Beacom:2019pzs, Zhou:2019vxt, Cooper_Sarkar_2011}, and $\bar{\lambda}_{GR}^{e}$ is the mean free path due to the Glashow Resonance $1/\left[ N_{e}(x)\sigma^{e}_{GR}(E) \right]$~\cite{PhysRev.118.316}. 

We also account for tau regeneration, neutrino-antineutrino coupling, and low-energy neutrino re-injection from neutral current interactions following the functional forms for neutrinos $F$ and antineutrinos $\bar{F}$ given by 
\begin{equation}\begin{split}
    F\left[\rho,\bar{\rho}, E, x\right] &= \sum\limits_{\alpha} \Pi_{\alpha}(E,x)\int\limits_{E}^{\infty}\dfrac{\text{Tr}\left[ \Pi(E_{\nu_{\alpha}},x)\rho(E_{\nu_{\alpha}}, x) \right]}{\lambda_{NC}^{\alpha}(E_{\nu_{\alpha}}, x) } \dfrac{\partial N_{NC}^{\alpha}(E_{\nu_{\alpha}}, E)}{\partial E} dE_{\nu_{\alpha}} \\ 
    &+\Pi_{\tau}(E,x) \int\limits_{E}^{\infty}\int\limits{E_{\tau}}^{\infty} \dfrac{\text{Tr}\left[ \Pi(E_{\nu_{\tau}},x)\rho(E_{\nu_{\tau}}, x) \right]}{\lambda_{NC}^{\tau}(E_{\nu_{\tau}}, E) } \dfrac{\partial N_{NC}^{\tau}(E_{\nu_{\tau}}, x)}{\partial E} \\
    &\times \dfrac{\partial N_{dec}^{all}(E_{\tau}, E)}{\partial E} dE_{\nu_{\tau}}dE_{\tau} \\
    &+\left[ \text{Br}\left(\tau^{-} \to \bar{\nu}_{e}\right)\Pi_{e}(E,x)\int\limits_{E}^{\infty}\int\limits_{E_{\tau}}^{\infty} \dfrac{\text{Tr}\left[ \bar{\Pi}(E_{\nu_{\tau}},x)\bar{\rho}(E_{\nu_{\tau}}, x) \right]}{\bar{\lambda}_{NC}^{\tau}(E_{\bar{\nu}_{\tau}}, x) }\right.  \\
    &+ \left. \text{Br}\left(\tau^{-} \to \bar{\nu}_{\mu}\right)\Pi_{\mu}(E,x)\int\limits_{E}^{\infty}\int\limits_{E_{\tau}}^{\infty} \dfrac{\text{Tr}\left[ \bar{\Pi}(E_{\nu_{\tau}},x)\bar{\rho}(E_{\nu_{\tau}}, x) \right]}{\bar{\lambda}_{NC}^{\tau}(E_{\bar{\nu}_{\tau}}, x) } \right] \\
    &\times \dfrac{\partial \bar{N}_{CC}^{\tau}(E_{\bar{\nu}_{\tau}}, E)}{\partial E} \dfrac{\partial\bar{N}_{dec}^{lep} (E_{\tau}, E)}{\partial E} dE_{\bar{\nu}_{\tau}} dE_{\tau}
\end{split}\end{equation}
and 
\begin{equation}\begin{split}
    \bar{F}\left[\rho,\bar{\rho}, E, x\right] &= \sum\limits_{\alpha}\bar{\Pi}_{\alpha}(E,x)\int\limits_{E}^{\infty}\dfrac{\text{Tr}\left[ \bar{\Pi}(E_{\bar{\nu}_{\alpha}}, x) \bar{\rho}(E_{\bar{\nu}_{\alpha}}, x) \right]}{\bar{\lambda}_{NC}^{\alpha}(E_{\bar{\nu}_{\alpha} }, x)}\dfrac{\partial\bar{N}_{NC}^{\alpha}(E_{\bar{\nu}_{\alpha}}, E)}{\partial E} d E_{\bar{\nu}_{\alpha}} \\ 
    &+ \bar{\Pi}_{\tau} (E,x) \int\limits_{E}^{\infty}\int\limits_{E_{\tau}}^{\infty}\dfrac{\text{Tr}\left[ \bar{\Pi}(E_{\bar{\nu}_{\tau}}, x) \bar{\rho}(E_{\bar{\nu}_{\tau}}, x) \right]}{\bar{\lambda}_{NC}^{\tau}(E_{\bar{\nu}_{\tau} }, x)}\dfrac{\partial\bar{N}_{NC}^{\tau}(E_{\bar{\nu}_{\tau}}, E)}{\partial E} \\
    &\times \dfrac{\partial\bar{N}_{dec}^{all}(E_{\tau}, E)}{\partial E} dE_{\bar{\nu}_{\tau}} dE_{\tau} \\
    &+ \left[ \text{Br}\left(\tau^{+} \to \nu_{e}\right)\bar{\Pi}_{e}(E,x)\int\limits_{E}^{\infty}\int\limits_{E_{\tau}}^{\infty} \dfrac{\text{Tr}\left[\Pi(E_{\nu_{\tau}}, x)\rho(E_{\nu_{\tau}}, x) \right] }{\lambda_{NC}^{\tau}(E_{\nu_{\tau}}, x)}  \right. \\
    &+ \left.\text{Br}\left(\tau^{+} \to \nu_{\mu}\right)\bar{\Pi}_{\mu}(E,x)\int\limits_{E}^{\infty}\int\limits_{E_{\tau}}^{\infty} \dfrac{\text{Tr}\left[\Pi(E_{\nu_{\tau}}, x)\rho(E_{\nu_{\tau}}, x) \right] }{\lambda_{NC}^{\tau}(E_{\nu_{\tau}}, x)}  \right] \\
    &\times \dfrac{\partial N_{CC}^{\tau}(E_{\nu_{\tau}}, E)}{\partial E} \dfrac{\partial N_{dec}^{lep}(E_{\tau}, E)}{\partial E} dE_{\nu_{\tau}} dE_{\tau} \\
    &+\left(\sum\limits_{\alpha} \bar{\Pi}_{\alpha} (E,x)\right) \int\limits_{E}^{\infty} \dfrac{\text{Tr}\left[\bar{\Pi}_{e}(E_{\bar{\nu}_{e}}, x) \bar{\rho}(E_{\bar{\nu}_{e}}, x) \right]}{\bar{\lambda}_{GR}^{e} (E_{\bar{\nu}_{e}}, x)}  \\
    &\times \dfrac{\partial\bar{N}_{GR}^{e}(E_{\bar{\nu}_{e}}, E)}{\partial E} dE_{\bar{\nu}_{e}}
\end{split}\end{equation}

Interaction rates in these functionals for neutral current, charged current, and Glashow resonance interactions are given by 

\begin{align}
    \dfrac{\partial N_{NC(CC)}^{\tau} (E_{\nu_{\tau}}, E)}{\partial E} &= \dfrac{1}{\sigma_{NC(CC)}^{\alpha} (E_{\nu_{\alpha} })}\dfrac{\partial\sigma_{NC(CC)}^{\alpha}(E_{\nu_{\alpha}}, E_{\alpha})}{\partial E_{\alpha}}\hspace{0.5cm} \text{ and}, \\
    \dfrac{\bar{N}_{GR}^{e} (E_{\bar{\nu}_{e}}, E)}{\partial E} &= \dfrac{1}{\sigma_{GR}^{e}(E_{\bar{\nu}_{e}})} \dfrac{\partial\sigma_{GR}^{e}(E_{\bar{\nu}_{e}}, E_{e})}{\partial E}.
\end{align}
The tau decay distributions for the leptonic-only modes ($N_{dec}^{lep}$) and for all-modes ($N_{dec}^{all}$) are given by 
\begin{align}
    \dfrac{\partial N_{dec}^{lep}(E_{\tau}, E)}{\partial E} &= \dfrac{1}{\bar{\Gamma}_{lep}^{\tau}(E_{\tau})} \dfrac{\partial\bar{\Gamma}_{lep}^{\tau}(E_{\tau},E)}{\partial E} \\
    \dfrac{\partial N_{dec}^{all}(E_{\tau}, E)}{\partial E} &= \dfrac{1}{\bar{\Gamma}_{all}^{\tau}(E_{\tau})} \dfrac{\partial\bar{\Gamma}_{all}^{\tau}(E_{\tau},E)}{\partial E} 
\end{align}

These systems are implemented within the \texttt{nuSQuIDS} framework~\cite{arguelles:2015nu, arguelles2021nusquids}, which implements Equations~\eqref{eq:nu_evol}-\eqref{eq:nubar_evol} to numerically propagate the state-density matrix along a neutrino fluxes' baseline. 
Different neutrino oscillation parameters can be specified, along with various other new physics scenarios, if desired.
\end{document}
