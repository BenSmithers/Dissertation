\documentclass[main.tex]{subfiles}

% \externalcitedocument{bibfile}

\begin{document}

\section{Photon Propagation}

After the event selection generated particles are propagated in-ice using PROPOSAL~\cite{Koehne:2013gpa} and photon propagation is simulated using CLSim. 

\section{Detector Simulation}

\section{Filters}

\subsection{Level 1}

The online filter

\subsection{Level 2}

The offline filter 

\subsection{Level 3 - Cascades}

The level 3 cascade filter is used to reduce the large number of triggered background events upon which more complicated event reconstructions must be run. 
The first step is to identify single events and coincident events; only single events are kept. 
Large numbers of cosmic ray air showers yield large quantities of down-going muons; by removing any events with topologically separated features, we cut down on background cosmic ray muons tremendously. 

Background muon events also enter into the detector from above, while the signal cascades start from within.
The next step for the cascades filter is to then identify which events are fully contained within the detector and which are not; uncontained (through-going) events are discarded. 
This also tremendously reduces the background in the sample. 
The cascades reconstruction algorithm, Monopod, is then run on the events that survive for use in future cuts. 

\end{document}