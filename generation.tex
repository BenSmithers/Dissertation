\documentclass[main.tex]{subfiles}

% \externalcitedocument{bibfile}

\begin{document}

\section{Photon Propagation}

After the event selection, which is described at-length in Chapter~\ref{chapter:LI}, generated particles are propagated in-ice using PROPOSAL~\cite{Koehne:2013gpa}. 

\section{Detector Simulation}

\section{Online, Offline Filters}

\subsection{Level 1}

The online filter

\subsection{Level 2}

The offline filter 


\section{Cascade Event Selection}

\subsection{Topological Splitting}

Before further processing and reconstruction is done on events, filter is applied to reduce the cosmic ray muon background while maintaining atmospheric signals as efficiently as possible. 
The first step of this process implements a topological triggering which distinguishes events between coincident and non-coincident events; the later of which are discarded. 
After this, events are classified as either contained within the detector or ``through going." For the cascades sample only those events contained within the detector are considered. 

Contained events must satisfy several criteria:
\begin{itemize}
    \item The depth of the first detected light must be greater than 70 meters from the outer edge of IceCube
    \item The first detected light must not be the outer layer of IceCube
\end{itemize}

\begin{enumerate}
    \item The number of topological splits must be one; coincident events are discarded
    \item The event must be contained 
\end{enumerate}

\subsection{Final Level Filtering}

This final level event filter was developed and published in Ref~\cite{2018PhDT17N}.

The monopod reconstruction algorithm is then run on events which survive the topological splitting cuts. 
This algorithm, in addition to calculating a reconstructed energy, interaction vertex, and direction, calculates a  reduced likelihood for the fit; the smaller the reduced log-likelihood is the better. 
Events for which the reduced algorithm fail, or are poor quality, are discarded. 
A poor quality reconstruction is one with a reduced log-likelihood of greater than 9.1.
Events within the IceCube dust-layer must have either 

Strict containment based on the reconstruction of the vertex coordinates from Monopod. 

Tighter coordinates cut at thetop and bottom of the detector. For those just outside an inner, fiducial, volume, we keep events only if they are well-reconstructed with a reduced log-likelihood<7.5. 

Finally, we cut those events with very large, negative, photon delay times. Such large delay times are common for detector triggers resulting from uncorrelated bundles of atmospheric cosmic ray muons. 



\end{document}